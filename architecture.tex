%%%%%%%%%%%%%%%%%%%%%%%%%%%%%%%%%%%%%%%%%%%%%%%%%%%%%%%%%%%%%%%%%%%%%%%%%%%%%%%%
\chapter{АНАЛИЗ СЕТЕВОЙ ИНФРАСТРУКТУРЫ}
%%%%%%%%%%%%%%%%%%%%%%%%%%%%%%%%%%%%%%%%%%%%%%%%%%%%%%%%%%%%%%%%%%%%%%%%%%%%%%%%

%%%%%%%%%%%%%%%%%%%%%%%%%%%%%%%%%%%%%%%%%%%%%%%%%%%%%%%%%%%%%%%%%%%%%%%%%%%%%%%%
\section{Обзор компонентов, необходимых для системы таргетирования рекламы.}
%%%%%%%%%%%%%%%%%%%%%%%%%%%%%%%%%%%%%%%%%%%%%%%%%%%%%%%%%%%%%%%%%%%%%%%%%%%%%%%%

В предыдущих главах представлен обзор одного компонента всей системы таргетирования рекламы. Этот компонент выполняет несколько функций:

\begin{itemize}
	\item предоставляет пользователю возможность формировать правила для генерации рекламных событий;
	\item генерирует события исходя из правил и имеющихся данных;
	\item получает информацию о новых клиентах системы и регистрирует их;
	\item получает и сохраняет информацию о местах, посещенных клиентами;
	\item регистрирует новых пользователей системы.
\end{itemize}

Для того чтобы рассмотренный ранее компонент работал корректно, ему необходимы еще два дополнительных средства:

\begin{itemize}
	\item компонент регистрации новых клиентов;
	\item Wi-Fi сканер.
\end{itemize}

\textbf{Компонент регистрации новых клиентов.}

Данный модуль является открытой Wi-Fi точкой доступа, которая установлена в помещении пользователя системы. Если организация пользователя работает в нескольких местах, то и точки доступа могут быть расположены в каждой из них. Если точка доступа установлена в помещении пользователя, то клиенты получают возможность бесплатно пользоваться интернетом. Это является хорошим стимулом для подключения к открытой сети. 

Регистрация клиента происходит при первом его подключении к открытому Wi-Fi. Для получения доступа в интернет клиент должен зарегистрироваться в системе. При регистрации запрашиваются такие данные как номер телефона, имя, возраст и пол человека, а также согласие на использование предоставленной информации. Полученная информация регистрируется на сервере и в дальнейшем используется для формирования рекламных сообщений.

Если клиент уже был ранее зарегистрирован в сети, то ему сразу предоставляется доступ в интернет без необходимости проходить процесс регистрации.

\textbf{Wi-Fi сканер.}

Wi-Fi сканер также работает по стандарту 802.11. Однако его задача - прослушивание эфира и выявление новых устройств, которые попадают в зону его досягаемости. Определив нового посетителя он формирует информацию о его идентификаторе, времени его появления и уходе из зоны досягаемости. После выхода посетителя из зоны сканирования, рассматриваемый компонент отправляет информацию на сервер, где она заносится в базу данных и в дальнейшем будет использована при формировании рекламных событий.

Wi-Fi сканер может быть установлен в любом месте и их количество может быть неограничено. Оптимальным вариантом являются места, где пользователю интересен факт появления и время появления его клиентов.

Далее необходимо рассмотреть некоторые вопросы связанные с работой двух описанных выше компонентов.

%%%%%%%%%%%%%%%%%%%%%%%%%%%%%%%%%%%%%%%%%%%%%%%%%%%%%%%%%%%%%%%%%%%%%%%%%%%%%%%%
\section{Обзор технологии Captive Portal для использования в целевой рекламе.}
%%%%%%%%%%%%%%%%%%%%%%%%%%%%%%%%%%%%%%%%%%%%%%%%%%%%%%%%%%%%%%%%%%%%%%%%%%%%%%%%

Данный пункт относится к компоненту регистрации новых клиентов. Для того чтобы зарегистрировать клиента, его необходимо при первом же подключении перенаправлять на сервер регистрации. Для таких задач существует технология Captive Portal. Рассмотрим ее более детально.

Работа Captive Portal может базироваться на двух подходах:

\begin{itemize}
	\item перенаправление за счет DNS запросов;
	\item перенаправление внутренними средствами маршрутизации.
\end{itemize}

При первом варианте DNS сервер настраивается таким образом, что на все запросы нового клиента ответ содержит адрес сервера регистрации. После регистрации клиента, его устройство начинает получать корректные ответы от DNS сервера. Такая системы может быть лекго обманута при статической настройке DNS сервера на устройстве клиента. Исключением является случай, когда сетевой экран настроен так, чтобы устранить возможность работы от стороннего DNS.

Второй вариант работы более надежен и подразумевает настройку маршрутизации таким образом, что все запросы по протоколу HTTP или HTTPS перенаправляются на локальный сервер. Captive Portal, организованный на платформе *nix может быть настроен с использованием утилиты iptables. В таком случае, после регистрации клиента в системе, для MAC адреса его устройства добавляется исключение, позволяющее ему использовать интернет.

Сценарий работы технологии Captive Portal существует достаточно давно, что делает его известным и простым в использовании. Данная технология позволяет доносить до клиентов дополнительную рекламу, что выгодно для любого бизнеса. Также, для Российской федерации введен закон, запрещающий наличие открытой сети без регистрации. Это означает что любая компания, предоставляя бесплатный интернет, не только может регистрировать клиентов, но и обязана это делать. Эти факторы делают технологию Captive Portal популярной и распространенной. 

Популярность сервиса привела к тому, что в устройства, способные подключаться к Wi-Fi, добавляется специальная функциональность, позволяющая проще и быстрее совершать регистрацию в сети. 

В ОС Android старше версии 4, мобильное устройство запрашивает с одного из серверов Google файл с названием generate\_204. В ответ оно ожидает код 204 (No Content). При отсутствии в ответе этого кода создается уведомление, по клику на которое запускается браузер со страницей из ответа. Поведение ОС Android может различаться для разных производителей. В некоторых вариантах вместо уведомления может сразу быть открыта страница авторизации. Так же различен и сам браузер - это может быть как стандартный браузер системы, так и специально созданной для этих целей простейший браузер.

В операционной системе iOS устройства также запрашивают файл с одного из сайтов Apple и сверяют его содержимое с ожидаемым. В случае обнаружения различий запускается утилита Captive Network Assistant, представляющий собой браузер без поддержки HTTP cookies.

Рассмотрение сервиса Captive Portal позволяет понять, что его функциональность полностью подходит для решаемой задачи. Рассмотренная технология позволяет регистрировать каждого нового клиента в системе, при этом запрашивая у него всю информацию, которая необходима для работы системы таргетирования рекламы, и которой клиент согласен поделиться.


%%%%%%%%%%%%%%%%%%%%%%%%%%%%%%%%%%%%%%%%%%%%%%%%%%%%%%%%%%%%%%%%%%%%%%%%%%%%%%%%
\section{Возможности стандарта IEEE 802.11 для выявления клиентов в зоне действия модуля Wi-Fi.}
%%%%%%%%%%%%%%%%%%%%%%%%%%%%%%%%%%%%%%%%%%%%%%%%%%%%%%%%%%%%%%%%%%%%%%%%%%%%%%%%

Задачей Wi-Fi сканера является выявление новых устройств в зоне действия Wi-Fi модуля и последующая регистрация этого события. Для того чтобы корректно выявлять новые устройства необходимо понимать работу протокола 802.11 и знать какие типы сообщений в нем используются.

IEEE 802.11 - это стандарт, описывающий коммуникацию оборудования по беспроводной сети в частотных диапазонах 0,9; 2,4; 3,6; 5 и 60 ГГц\cite{iee802_11}. Стандарт стандарт состоит из набора других стандартов ориентирующихся на конкретные вопросы. Например стандарт 802.11i описывает безопасность связи, а 802.11n вариант коммуникаций со скоростью до 600 МБит/с. Стандарт описывает работу на канальном уровне модели OSI.

Стандарт 802.11 описывает три типа фреймов

\begin{enumerate}
	\item Фреймы Управления (Management frames).
	\item Фреймы Контроля (Control frames).
	\item Фреймы Данных (Data frames).
\end{enumerate}

Внутри каждого фрейма имеются поля, которые определяют версию протокола тип фрейма и различные идентификаторы. Также каждый фрейм несет в себе информацию о MAC-адресе отправителя и MAC-адресе получателя.

\textbf{Фреймы Управления (Management Frames)}

Фреймы управления используются для установления и поддержания канала общения между коммутирующими устройствами.

Всего стандарт 802.11 определяет 14 типов фреймов управления:

\begin{enumerate}
	\item Association request,
	\item Association response,
	\item Reassociation request,
	\item Reassociation response,
	\item Probe request,
	\item Probe response,
	\item Beacon,
	\item ATIM (Announcement traffica indication mesage),
	\item Disassiciation,
	\item Authentication,
	\item Deauthentication,
	\item Action,
	\item Action No Ack,
	\item Timing advertisement
\end{enumerate}

Рассмотрим некоторые фреймы.

\textit{Фрейм аутентификации} (Authentication frame) отправляется устройством, желающим подключиться к точке доступа. В фрейме присутствует идентификационная информация об устройстве. Точка доступа может принять идентификацию или отвергнуть.

\textit{Фрейм ассоциации} (Association request frame). Этот фрейм отправляется от устройства к точке доступа. Данным фреймом устройство просит точку доступа зарезервировать ресурсы для будущей сессии. Внутри фрейма содержится информация о характеристиках канала, таких как скорость передачи, с которыми может работать устройство. Для идентификации точки доступа, в данном фрейме отправляется SSID(имя) сети. Если точка доступа принимает запрос ассоциации, то она отвечает устройству фреймом ответа на ассоциацию, внутри которого хранится идентификатор ассоциации.

\textit{Фрейм - маячок} (Beacon frame) периодически отправляется точками доступа и содержит информацию, такую как SSID точки доступа, частотный канал, временные маркеры для синхронизации времени, поддерживаемые скорости и другое. Этот фрейм позволяет всем устройствам определять досягаемые точки доступа и каналы на которых они работают.

\textit{Фрейм - проба} (Probe request frame) отправляется мобильными устройствами, для установления размещенных в зоне покрытия точек доступа. Запрос может быть как широковещательным, так и с указанием конкретных SSID точек доступа. В ответном фрейме точка доступа отправляет информацию о функциональности, поддерживаемых скоростях и другое.

Подключение Wi-Fi устройства к точке доступа происходит несколько этапов. Первым этапом является этап аутентификации, затем этап ассоциации. Если точка доступа защищена паролем, то третьим этапом является проверка пароля, а при наличии шифрования и установление его характеристик.

\textbf{Фреймы Контроля} (Control Frames)

Фреймы контроля позволяют доставлять данные между Wi-Fi устройствами. В стандарте 802.11 определено 9 типов фреймов контроля. Однако в отличие от фреймов управления они не несут какой-либо полезной информации в условиях нашей задачи.

\textbf{Фреймы Данных Wi-Fi}

Главной задачей протокола является передача информации протоколов вышестоящего уровня. Для этого и предназначены 15 типов фреймов данных, описанных в стандарте 802.11. Фреймы данных используются уже при установившемся соединении, из чего следует, что для задачи выявления новых устройств они не интересны.

Исходя из вышесказанного, следует, что наиболее интересным является управляющий фрейм Probe Request. Он несет в себе информацию о MAC-адресе мобильного устройства и отправляется постоянно при поиске доступных Wi-Fi сетей. Таким образом, устройство, пронесенное мимо Wi-Fi сканера со включенным модулем Wi-Fi будет рассылать запросы на поиск сетей и попадет в список зафиксированных устройств. Это позволит определить факт посещения клиентом определенного места.


%%%%%%%%%%%%%%%%%%%%%%%%%%%%%%%%%%%%%%%%%%%%%%%%%%%%%%%%%%%%%%%%%%%%%%%%%%%%%%%%
\section{Резюме}
%%%%%%%%%%%%%%%%%%%%%%%%%%%%%%%%%%%%%%%%%%%%%%%%%%%%%%%%%%%%%%%%%%%%%%%%%%%%%%%%

В данной главе рассмотрены все оставшиеся компоненты системы формирования целевой рекламы. Таким образом всего в системе три компонента: компонент регистрации новых пользователей, Wi-Fi сканер и сервер с интеллектуальной системой для формирования рекламных событий. Как было установлено в главе про облачные системы, оптимальным протоколом взаимодействия между компонентами является HTTPS протокол. Дальнейшим этапом станет подбор средств разработки и сама разработка описанных компонентов.
