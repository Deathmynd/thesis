%%%%%%%%%%%%%%%%%%%%%%%%%%%%%%%%%%%%%%%%%%%%%%%%%%%%%%%%%%%%%%%%%%%%%%%%%%%%%%%%
\conclusion
%%%%%%%%%%%%%%%%%%%%%%%%%%%%%%%%%%%%%%%%%%%%%%%%%%%%%%%%%%%%%%%%%%%%%%%%%%%%%%%%

Целевая реклама в будущем может выйти на новый уровень, позволяющий использовать так называемый геоповеденческий таргетинг. Он позволит выбирать целевую аудиторию исходя из информации о посещенных человеком мест. В данной работе реализована система, позволяющая воплотить идею геоповеденческого таргетинга.

В работе выполнено рассмотрение видов таргетированной рекламы и информации, которую они использует. Проанализирован вопрос законности сбора и использования персональных данных в маркетинговых целях. Составлено пользовательское соглашений, принимаемое клиентами во время регистрации.

Рассмотрены виды интеллектуальных систем, их основные компоненты и модели реализации. На основе полученной информации выделен тип ИИ - экспертные системы, как наиболее подходящий в целях формирования рекламных сообщений. С использованием типовой структуры ЭС спроектирован компонент генерации рекламных событий.

В качестве платформы для размещения компонента генерации рекламы выбрана SaaS технология. Предварительно рассмотрены преимущества и недостатки облачных систем. Проанализированы угрозы данным как на сервере облачных технологий, так и в каналах связи. Для защиты данных в каналах связи решено использовать протокол HTTPS.

В результате была разработана система, состоящая из трех компонентов: сервера регистрации новых клиентов для пополнения клиентской базы данных, Wi-Fi сканера, позволяющего получать данные о географических перемещениях клиентов и системы генерации рекламных сообщений. Последний компонент в своем составе так же имеет веб-интерфейс для формирования правил, являющихся базой знаний для генератора событий.

Проведенное тестирование показало работоспособность каждого компонента в отдельности и системы в целом. Дальнейшее развитие

\blindtext
