%%%%%%%%%%%%%%%%%%%%%%%%%%%%%%%%%%%%%%%%%%%%%%%%%%%%%%%%%%%%%%%%%%%%%%%%%%%%%%%%
\chapter{РОЛЬ И МЕСТО ТЕХНОЛОГИЙ В ЦЕЛЕВОЙ РЕКЛАМЕ}
%%%%%%%%%%%%%%%%%%%%%%%%%%%%%%%%%%%%%%%%%%%%%%%%%%%%%%%%%%%%%%%%%%%%%%%%%%%%%%%%

Таргетированная реклама - это рекламные текстовые сообщения, дополненные изображениями, анимацией или видео роликами и демонстрирующиеся только  лицам, которые удовлетворяют критериям рекламодателя.

%%%%%%%%%%%%%%%%%%%%%%%%%%%%%%%%%%%%%%%%%%%%%%%%%%%%%%%%%%%%%%%%%%%%%%%%%%%%%%%%
\section{Развития целевой рекламы}
%%%%%%%%%%%%%%%%%%%%%%%%%%%%%%%%%%%%%%%%%%%%%%%%%%%%%%%%%%%%%%%%%%%%%%%%%%%%%%%%

Первое онлайн объявление создано в 1994 году, когда интернетом пользовалось всего 30 миллионов человек. В веб версии магазина HotWired 27 октября 1994 года было впервые запущено баннерное обьявление для компании  AT\&T.\cite{kumar2016evolution}

Впервые интернет реклама, имеющая целевой характер, появилась в поисковой системе goto.com в 1998 году. В настоящее время компания известна под названием Overture и принадлежит Yahoo!. Идея заключалась в том, чтобы продавать ссылки, которые видит пользователь в результатах поиска. Сами ссылки подбираются исходя из контекста поисковых запросов, введенных пользователем. Идея принадлежит американскому финансисту Биллу Гроссу, который является основателем компании Idealab.\cite{wang2016display} 

Такой подход к продвижению товаров и брендов был выгоден не только рекламодателям, но и создателям поисковых систем, которые могли получать от этого прибыль. Данный вид рекламы в дальнейшем был назван контекстной рекламой.

В дальнейшем идею контекстной рекламы стали реализовывать и другие компании. Например компания Google создала сервис Google AdWords, в который позже была добавлена возможность проведения торгов за рекламное место.

В настоящее время за рубежом лидерами в области контекстной рекламы являются такие компании как: Google AdWords, Yahoo! Search Marketing и Microsoft Advertising. В нашей стране широкую популярность имеет Яндекс Директ.

Сильное развитие таргетированные реклама получила с появлением социальных сетей. Именно в них начали использовать такую информацию о клиенте как возраст, пол, семейное положение, геолокацию, круг общения. 

В настоящее время информация о потенциальных клиентах собирается не только на сайтах с персональными данными, но и самими браузерами или техническими устройствами такими как мобильный телефон или умный телевизор. Например браузеру известная информацию о посещаемых пользователем сайтах и времени, в течении которого он был запущен. Собранная информация в дальнейшем используется сторонними компаниями, которым выгодно найти клиента с определенными характеристиками. 

Для того чтобы понять как работает и как формируется целевая реклама необходимо сначала разобраться в интернет рекламы в целом. Для этого рассмотрим виды на которые она разделяется.

%%%%%%%%%%%%%%%%%%%%%%%%%%%%%%%%%%%%%%%%%%%%%%%%%%%%%%%%%%%%%%%%%%%%%%%%%%%%%%%%
\section{Технологий в интернет рекламе}
%%%%%%%%%%%%%%%%%%%%%%%%%%%%%%%%%%%%%%%%%%%%%%%%%%%%%%%%%%%%%%%%%%%%%%%%%%%%%%%%

Рекламу в интернете разделяют на несколько видов. Критерием для разделения служит используемая технология, позволяющая тем или иным способом довести контент до клиента.

\textbf{Контекстная реклама}

Этот вид рекламы использует контекст просматриваемой страницы. Рекламные сообщения или баннеры подбирается исходя из ключевых слов, которым соответствует содержание посещаемого интернет ресурса. Этот вид рекламы можно увидеть на большинстве современных сайтов. Например, при посещении сайта по организации туристических походов можно увидеть рекламу магазина, продающего туристическое снаряжение.
    
\textbf{Маркетинг в поисковых системах}

Частным случаем контекстной рекламы являются рекламные ссылки в поисковых системах. Это маркетинговая практика, при которой рекламные объявления встраиваются в результаты поисковых запросов. Рекламодатель заранее указывает ключевые слова, которые будут использованы для выявления целевой аудитории. При выдаче клиенту результатов поиска в них добавляются рекламные сообщения, которые по ключевым словам соответствуют поисковому запросу.

При таком методе распространения рекламодатель оплачивает только ту рекламу, которой воспользовался клиент. Такой подход также называется “оплата по клику” . Рекламные сообщения могут быть как небольшими текстовыми фразами, так и более информативными блоками с описанием товара, его цены и ссылкой на обзор.

Главной особенностью поискового маркетинга является то, что он предоставляет рекламодателям возможность размещать свои объявления перед мотивированными клиентами, которые получают рекламу в тот момент, когда сами готовы совершать покупки. Никакой другой тип распространения рекламы не может обеспечить подобного. Поэтому маркетинг в поисковых системах является одним из самых эффективных видов целевой рекламы.

\textbf{Таргетинг в социальных сетях}

Это еще один канал интернет маркетинга. В социальных сетях пользователи сами оставляют о себе полезную для рекламодателя информацию, такую как возраст, пол, сфера деятельности, интересы. Такая информация позволяет достаточно точно подбирать целевые группы и определять где эти группы сосредоточены. Целевая аудитория, выделенная по множеству критериев, более лояльно отнесется к товару из близкой ей тематики.
    
\textbf{Нативная реклама}

Нативная реклама - это реклама завуалированная под развлекательную или обзорную статью. Такая статья имеет привычный читателю вид и своим заголовком обещает рассмотреть какой-либо вопрос. Однако на самом деле ей целью является реклама определенного товара. Как правило, такие статьи имеют шаблонные заголовки: "Как выбрать хороший Х", "Топ 5 Х" или "Обзор популярных Х". Данный тип рекламы является относительно дорогим, но при этом и достаточно эффективным.   

Все выше изложенные способы доведения рекламы до клиента используют определенную информацию для выделения групп клиентов, заинтересованных в том или ином товаре. Таким образом следует рассмотреть какая информация может быть полезной для формирования клиентской базы. В дальнейшем клиентская база позволит выделять целевые объекты по их интересам.
    
%%%%%%%%%%%%%%%%%%%%%%%%%%%%%%%%%%%%%%%%%%%%%%%%%%%%%%%%%%%%%%%%%%%%%%%%%%%%%%%%
\section{Виды целевой рекламы по типу используемой информации} 
%%%%%%%%%%%%%%%%%%%%%%%%%%%%%%%%%%%%%%%%%%%%%%%%%%%%%%%%%%%%%%%%%%%%%%%%%%%%%%%%   
    
Рассмотрим виды целевой рекламы, которые разделены по типу информация, используемой для формирования целевой аудитории.

\textbf{Таргетинг по интересам}
Таргетинг по интересам подразумевает показ рекламы, соответствующей интересам целевого объекта. Интересы клиента могут быть установлены, например, по тематике посещаемой в данные момент web страницы.

\textbf{Временной таргетинг}

Временной таргетинг позволяет рекламодателям уточнять время, в которое будут задействованы их рекламные сообщения. Чаще всего время привязано к локальному времени конечного клиента. Это позволяет использовать рекламу более эффективно, т.к. сокращает число показов в часы, когда потенциальные клиенты не пользуются интернет ресурсами. \cite{timeTargeting1}
    
\textbf{Демографический таргетинг}

Данный вид маркетинга использует в работе демографическую информацию, такую как пол, возраст, доход, семейное положение и другое. Эти данные позволяют значительно сократить целевую аудиторию и рекламировать товар, только тем людям, которые потенциально могут быть в нем заинтересованы. В большинстве случаев такая информация может быть собрана при заполнении регистрационной формы на сайте рекламодателя.

\textbf{Географический таргетинг}

Географический таргетинг позволяет формировать группы клиентов исходя из их местоположения. Критериями такой рекламы может быть страна,  регион, город или почтовый индекс. Например если компания ведет свою деятельность только на территории одного города, то для нее не имеет смысл рекламировать свои услуги в других городах. Информация для географического таргетинга может быть получена как при регистрации клиента на сайте рекламодателя, так и из других источников.\cite{kindOfTarget1}

\textbf{Поведенческий таргетинг}

Данный вид формирования рекламы основан на интересах посетителя.Эти интересы устанавливаются исходя из поведения пользователя в браузере, ранее просмотренного контента, выполненного поиска, посещенных сайтов. Например, если установлено, что пользователь ранее просматривал туристические туры, то в рекламных баннерах могут быть предложения на покупку авиа или жд билетов. 
        
\textbf{Геоповеденческий таргетинг}

Исходными данными для геоповеденческого таргетинга является информация о перемещениях клиента и его остановках. Используя эту информацию можно точно определить привычки и пристрастия объекта. Например, если клиент часто посещает магазины со спорттоварами, то и реклама компаний, занимающихся распространением спортивной амуниции его заинтересует.

Перечисленные виды таргетинга подразумевают сбор и использование определенного вида информации о клиенте. Использование какой-то части такой информации может быть запрещено законодательством РФ. Это значит что прежде чем создавать собственную систему формирования целевой рекламы, необходимо разобраться с вопросом о законности хранения и использования видов данных, которые будут использованы в создаваемой системе.
        
%%%%%%%%%%%%%%%%%%%%%%%%%%%%%%%%%%%%%%%%%%%%%%%%%%%%%%%%%%%%%%%%%%%%%%%%%%%%%%%%
\section{Юридический аспект. Проблема хранения и использования персональных данных}
%%%%%%%%%%%%%%%%%%%%%%%%%%%%%%%%%%%%%%%%%%%%%%%%%%%%%%%%%%%%%%%%%%%%%%%%%%%%%%%%

%%%%%%%%%%%%%%%%%%%%%%%%%%%%%%%%%%%%%%%%%%%%%%%%%%%%%%%%%%%%%%%%%%%%%%%%%%%%%%%%
\section{Резюме}
%%%%%%%%%%%%%%%%%%%%%%%%%%%%%%%%%%%%%%%%%%%%%%%%%%%%%%%%%%%%%%%%%%%%%%%%%%%%%%%%

В данной главе рассмотрены основные методы реализации рекламы в интернете. Рассмотрены виды информации, которая позволяет осуществлять выбор целевой аудитории в задачах таргетированной рекламы. Проанализирован юридический аспект в вопросе хранения и использования клиентской информации. Дальнейшим этапом станет рассмотрение интеллектуальной системы, как системы для формирования целевой рекламы на основе клиентской базы данных.