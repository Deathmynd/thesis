%%%%%%%%%%%%%%%%%%%%%%%%%%%%%%%%%%%%%%%%%%%%%%%%%%%%%%%%%%%%%%%%%%%%%%%%%%%%%%%%
\intro
%%%%%%%%%%%%%%%%%%%%%%%%%%%%%%%%%%%%%%%%%%%%%%%%%%%%%%%%%%%%%%%%%%%%%%%%%%%%%%%%

В современном мире реклама встречается практически на каждом шагу. Ее показывают по телевизору, распространяют листовками, почтовыми сообщениями, развешивают по городу в виде банеров. Существует реклама и в сети интернет. Однако, в отличии от обычной рекламы, у интернет рекламы есть одно большое преимущество: она может быть целенаправленной. Это значит, что рекламные сообщения, которые увидит пользователь, не будут выбраны случайным образом из общего числа заявок, а будут использовать специальный алгоритм, позволяющий конкретизировать аудиторию. Целевая или таргетированная реклама становится возможной благодаря тому, что интернет в буквальном смысле запоминает все проделанные человеком действия. Будь то посещение тематических сайтов, совершение покупок в интернет магазине или просмотр видео --- все действия остаются в видет информации на серверах компаний и при необходимости могут быть использованы для выявления предпочтений человека. Таргетированной рекламой целенаправленно занимаются такие компании как Google, Яндекс и Microsoft. Спрос на такой вид услуг никогда не иссякнет, т.к. любой компании выгодно привлечь именно ту аудиторию, которую заинтересует их услуга или товар.

Дальнейшее развити целевой рекламы --- это вынос ее возможностей за пределы интернета. Возможность привлечь посетителя, который просто проходит каждый день мимо вашего заведения --- это уже новый уровень таргетинга. Сделать это можно на основе данных, которые вещаются в эфир мобильным устройством потенциального клиента. К таким данным, например, относится информация от включенного Wi-Fi модуля, который есть в каждом современном смартфоне. 

Целью данной работы является проектирование и разработка системы, которая позволила бы пользователям составлять правила таргетирования рекламы, на основе данных о посещении клиентами определенных географических мест. Для создания клиентской базы используется общественная Wi-Fi сеть с открытым доступом и требующая регистрации клиентов. А для формирования информации о географических точках, посещенных клиентом --- специальные комплексы, включающие в себя Wi-Fi модуль и позволяющие выявлять проносимые мимо устройства.

Работа состоит из шести разделов. В первом разделе рассматриваются вопросы связанные с таргетированной рекламой в целом. Рассматриваются уже существующие методы таргетирования и виды используемой информации. Решается вопрос о законности использования персональных данных.

Во втором разделе осуществляется обзор видов интеллектуальных систем и их типовая структура. Определяется конфигурация приложения, которая может быть использована в целях таргетирования рекламы.

Третий раздел посвящен облачным вычислениям и их видам. Рассматривается вопрос о размещении компонента генерации рекламных сообщений в облачной инфраструктуре.

В четвертом разделе описывается общая архитектура системы. Рассматриваются технология Captive Portal для реализации компонента регистрации пользователей. В разделе описываются возможности стандарта 802.11, позволяющие выявить новые устройства в зоне действия Wi-Fi модуля. 

Пятый раздел посвящен проектированию и разработке каждого компонента. В разделе приведены архитектуры всех компонентов, описываются используемые при разработке инструменты и библиотеки.

Раздел номер шесть посвящен тестированию разработанного прототипа. Осуществляется как тестирование каждого компонента по отдельности, так и целостное тестирование системы.
